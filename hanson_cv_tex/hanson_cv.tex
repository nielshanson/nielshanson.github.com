%%%%%%%%%%%%%%%%%%%%%%%%%%%%%%%%%%%%%%%%%
% Friggeri Resume/CV
% XeLaTeX Template
% Version 1.1 (9/2/15)
%
% This template has been downloaded from:
% http://www.LaTeXTemplates.com
%
% Original author:
% Adrien Friggeri (adrien@friggeri.net)
% https://github.com/afriggeri/CV
%
% License:
% CC BY-NC-SA 3.0 (http://creativecommons.org/licenses/by-nc-sa/3.0/)
%
% Important notes:
% This template needs to be compiled with XeLaTeX and the bibliography, if used,
% needs to be compiled with biber rather than bibtex.
%
%%%%%%%%%%%%%%%%%%%%%%%%%%%%%%%%%%%%%%%%%

\documentclass[print]{hanson_cv} % Add 'print' as an option into the square bracket to remove colors from this template for printing

% \addbibresource{bibliography.bib} % Specify the bibliography file to include publications

\begin{document}

\header{Niels}{Hanson}{Data Scientist and Software Developer} % Your name and current job title/field

%----------------------------------------------------------------------------------------
%	SIDEBAR SECTION
%----------------------------------------------------------------------------------------

\begin{aside} % In the aside, each new line forces a line break
\section{Contact}
4557 West 14th Ave.
Vancouver, B.C.
Canada
~
+1 (778) 385 4465
~
\href{mailto:nielshanson@gmail.com}{nielshanson@gmail.com}
\href{http://www.nielshanson.com}{nielshanson.com}
\href{http://www.github.com/nielshanson}{GitHub:nielshanson}
\section{Programming}
Python, R, Java, C++, JavaScript, PHP
\end{aside}

%----------------------------------------------------------------------------------------
%	EDUCATION SECTION
%----------------------------------------------------------------------------------------

\section{Education}

\begin{entrylist}
%------------------------------------------------
\entry
{2006--2011}
{B.Sc. {\normalfont  Computer Science (Statistics Minor)}}
{University of British Columbia}
{Courses: Algorithms, Software Development, Machine Learning, Databases, Statistical Inference, Experimental Design}
\entry
{2011--2015}
{Ph.D. {\normalfont  Bioinformatics}}
{University of British Columbia}
{Courses: Bioinformatics Algorithms, Machine Learning, Information Visualization, Numerical Methods}

%------------------------------------------------
\end{entrylist}

\section{Awards}

\begin{entrylist}
%------------------------------------------------
\entry
{2012}
{Four Year Doctoral Fellowship (4YF)}
{University of British Columbia}
{Awarded to incoming doctoral graduate students awarded on academic excellence, upon the recommendation of the graduate program.}
%------------------------------------------------
\end{entrylist}

%----------------------------------------------------------------------------------------
%	WORK EXPERIENCE SECTION
%----------------------------------------------------------------------------------------
\section{Experience}
\begin{entrylist}
%------------------------------------------------
\entry
{2011--Now}
{Steven J. Hallam Laboratory}
{University of British Columbia}
{\emph{Bioinformatics Ph.D. Candidate} \\
Primary research focuses on the development of MetaPathways, a pipeline developed in Python and C++ for the Big Data processing and analysis challenges of next-generation environmental sequence information. The software accommodates a wide range of experimental conditions, integrates with HPC resources via a master-worker model, and is controlled locally via user-friendly GUI. Moreover, I am involved in all areas of lab data analysis and research, and have utilized a combination of Python and R to develop bespoke solutions to a variety of analytical challenges.
}
\end{entrylist}
%------------------------------------------------
\begin{entrylist}
\entry
{2011}
{Microbiology and Immunology}
{University of British Columbia}
{\emph{Software Developer} \\
Developer of RameyDB, a database-driven web-application for the cataloging of bacterial strains and plasmids at the University of British Columbia. Utilized the LAMP stack, JavaScript, and PHP to create an easy-to-use web interface. Implemented a number of features to improve usability including an advance search form, user search histories, and intelligent autocomplete to ease data entry. }
\end{entrylist}
%------------------------------------------------
\begin{entrylist}
\entry
{2010}
{Evan E. Eichler Laboratory}
{University of Washington}
{\emph{Bioinformatics Intern} \\
Summer research position involved the development of a database-driven human genome visualization software for analysis of autistic genomes. Designed and implemented MySQL back-end database and a Java Swing-based GUI and visualization environment for genetic variants predicted via a Hidden Markov Model. Designing efficient layout algorithms allowed for the rapid analysis and validation of important genomic alterations, replacing a previously tedious and disconnected analysis pipeline.}
\end{entrylist}
%------------------------------------------------
\begin{entrylist}
\entry
{2009}
{Centre for Microbial Disease and Immunity Research}
{University of British Columbia}
{\emph{Software Developer} \\
Developed a data-base driven web-application PaIntDB, a biological database for the interactive query and exploration of protein-protein interaction data using Jakarta Struts and JavaScript with a LAMP stack back end. Performed multiple rounds of development and consultation with microbiologist and UBC and SFU stakeholders.}
%------------------------------------------------
\end{entrylist}
\begin{entrylist}
\entry
{2008}
{UBC Information Technology}
{University of British Columbia}
{\emph{Software Developer} \\
Key developer of a web-based events calendar UBCEvents. Designed, styled, and implemented the Web UI through CSS, HTML, and JavaScript. Developed and modified the underlying open source Bedework XSLT code to parse and translate XML objects retrieved from an Oracle database into structured webpages. Designed statistical scripts using Perl and R to parse logs and summarize calendar use.}
\end{entrylist}

\section{Publications}

For citations please see my Google Scholar page. \href{http://goo.gl/aYo3Wa}{http://goo.gl/aYo3Wa}

\begin{itemize}
\item
  \textbf{Niels W. Hanson}, Kishori M. Konwar, Shang-Ju Wu, Steven J. Hallam. \emph{Introduction to the Analysis of Environmental Sequence Information Using MetaPathways}, Computational Methods for Next Generation Sequencing Data Analysis. Book Chapter. Wiley Series in Bioinformatics. In Press.
\item
Kishori M. Konwar, \textbf{Niels W. Hanson}, Maya P. Bhatia, Dongjae Kim, Shang-Ju Wu, Aria S. Hahn, Connor Morgan-Lang, Hiu Kan Cheung, Steven J. Hallam. \emph{MetaPathways v2.5: Quantitative functional, taxonomic, and usability improvements}, Bioinformatics. In Press. 

\item
  Christopher E. Lawson, Blake J. Strachan, \textbf{Niels W. Hanson}, Aria S. Hahn, Eric R. Hall, Barry Rabinowitz, Donald S. Mavinic, William D. Ramey, Steven J. Hallam. \emph{Rare taxa have potential to make metabolic contributions in enhanced biological phosphorus removal ecosystems}, Environmental Microbiology. April 2015. \href{http://dx.doi.org/10.1111/1462-2920.12875}{doi:10.1111/1462-2920.12875} 
\item
  Dongjae Kim, Kishori M. Konwar, \textbf{Niels W. Hanson}, Steven J.
  Hallam. \emph{Koonkie: An Automated Software Tool for Processing
  Environmental Sequence Information using Hadoop}, ASE BigData 2014.
  Harvard University, December 14-16,
  2014.\\\href{http://www.ase360.org/handle/123456789/164}{http://www.ase360.org/handle/123456789/164}
\item
  \textbf{Niels W. Hanson}, Kishori M. Konwar, Alyse K. Hawley, Tomer
  Altman, Peter D. Karp, Steven J. Hallam. \emph{Metabolic pathways for
  the whole community}, BMC Genomics. July 2014.
  \href{http://dx.doi.org/10.1186/1471-2164-15-619}{doi:10.1186/1471-2164-15-619}
\item
  \textbf{Niels W. Hanson}, Kishori M. Konwar, Shang-Ju Wu, Steven J.
  Hallam. \emph{MetaPathways v2.0: A master-worker model for
  environmental Pathway/Genome Database construction on grids and
  clouds.} Proceedings of the 2014 IEEE Conference on Computational
  Intelligence in Bioinformatics and Computational Biology (CIBCB 2014),
  Honolulu, HI, USA, May 21-24, 2014.
  \href{http://ieeexplore.ieee.org/xpl/articleDetails.jsp?arnumber=6845516}{doi:10.1109/CIBCB.2014.6845516}
\item
  Jody J. Wright, Keith Mewis, \textbf{Niels W. Hanson}, Kishori M.
  Konwar, Kendra R. Maas, Steven J. Hallam. \emph{Genomic properties of
  Marine Group A bacteria indicate a role in the marine sulfur cycle},
  ISME Journal. September 2013.
  \href{http://dx.doi.org/10.1038/ismej.2013.152}{doi:10.1038/ismej.2013.152}
\item
  Dongshan An, Sean Michael Caffrey, Jung Soh, Akhil Agrawal, Damon
  Brown, Karen Budwill, Xiaoli Dong, Peter F. Dunfield, Julia Foght,
  Lisa M. Gieg, Steven J. Hallam, \textbf{Niels W. Hanson}, Zhiguo He,
  Thomas R. Jack, Jonathan Klassen, Kishori M. Konwar, Eugene Kuatsjah,
  Carmen Li, Steve Larter, Verlyn Leopatra, Camilla L Nesbo, Thomas B.P.
  Oldenburg, Antoine P. Pag\'{e}, Esther Ramos-Padron, Fauziah Rochman, Ali
  Saidi-Mehrabad, Christoph W. Sensen, Payal Sipahimalani, Young C.
  Song, Sandra Wilson, Gregor Wolbring, Ginny Wong, Gerritt Voordouw.
   \emph{Metagenomics of Hydrocarbon Resource Environments Indicates Aerobic
  Taxa and Genes to be Unexpectedly Common.} Environmental Science
  \& Technology, September 2013
  \href{http://dx.doi.org/10.1021/es4020184}{doi:10.1021/es4020184}
\item
  Brandon K. Swan, Ben Tupper, Alexander Sczyrba, Federico M. Lauro,
  Manuel Martinez-Garcia, Jos\'{e} M. Gonzalez, Haiwei Luo, Jody J. Wright,
  Zachary C. Landry, \textbf{Niels W. Hanson}, Brian P. Thompson, Nicole
  J. Poulton, Patrick Schwientek, Silvia G. Acinas, Stephen J
  Giovannoni, Mary Ann Moran, Steven J. Hallam, Ricardo Cavicchioli,
  Tanja Woyke, Ramunas Stepanauskas. \emph{Prevalent genome streamlining
  and latitudinal divergence of planktonic bacteria in the surface
  ocean}, Proceedings of the National Academy of Sciences. July 2013.
  \href{http://dx.doi.org/10.1073/pnas.1304246110}{doi:10.1073/pnas.1304246110}
\item
  Kishori M. Konwar, \textbf{Niels W. Hanson}, Antoine P. Pag\'{e}, Steven
  J. Hallam. \emph{MetaPathways: a modular pipeline for constructing
  pathway/genome databases from environmental sequence information}, BMC
  Bioinformatics. June 2013. \href{http://dx.doi.org/10.1186/1471-2105-14-202}{doi:10.1186/1471-2105-14-202}
\item
  W. Evan Durno, \textbf{Niels W. Hanson}, Kishori M. Konwar, Steven J.
  Hallam. \emph{Expanding the boundaries of local similarity analysis}, BMC Bioinformatics.
  February 2013. \href{http://dx.doi.org/10.1186/1471-2164-14-S1-S3}{doi:10.1186/1471-2164-14-S1-S3}
\item
  Kaston Leung, Hans Zahn, Timothy Leaver, Kishori M. Konwar,
  \textbf{Niels W. Hanson}, Antoine P. Pag\'{e}, Chien-Chi Lo, Patrick S.
  Chain, Steven J. Hallam, Carl L. Hansen. \emph{A programmable
  droplet-based microfluidic device applied to multiparameter analysis
  of single microbes and microbial communities}, Proceedings of the
  National Academy of Sciences. May 2012.
  \href{http://dx.doi.org/10.1073/pnas.1106752109}{doi:10.1073/pnas.1106752109}
\end{itemize}

\section{In-review}
\begin{itemize}
\item
  Dongjae Kim, Aria S. Hahn, Shang-Ju Wu, \textbf{Niels W. Hanson}, Kishori M. Konwar, Steven J. Hallam. \emph{FragGeneScan+: high-throughput short-read gene prediction}, BMC Bioinformatics. In review.
\item
  Aria S. Hahn, \textbf{Niels W. Hanson}, Dongjae Kim, Kishori M. Konwar, Steven J. Hallam. \emph{Assembly independent functional annotation of short-read data using SOFA: Short-ORF Functional Annotation}, 11th International Symposium on Bioinformatics Research and Applications (ISBRA 2015). In review.
\end{itemize}
%----------------------------------------------------------------------------------------
%	COMMUNICATION SKILLS SECTION
%----------------------------------------------------------------------------------------

\section{Conferences \& Talks}
\begin{itemize}
\item
  \textbf{Niels W. Hanson}, Kishori M. Konwar, Shang-Ju Wu, Steven J.
  Hallam. \emph{MetaPathways v2.0: A master-worker model for
  environmental Pathway/Genome Database construction on grids and
  clouds.} IEEE Conference on Computational Intelligence in
  Bioinformatics and Computational Biology (CIBCB 2014). Presentation.
  Honolulu, HI, USA. May 21--24 2013.
\item
  Frances K. Russell, \textbf{Niels W. Hanson}, Kishori M. Konwar,
  Steven J. Hallam. \emph{Hierarchical and High-Performance clustering
  and annotation for large protein sequence databases.} 
  Canadian High-Performance Computing Symposium (HPCS 2013). Poster. Ottawa,
  ON, Canada. June 2--6 2013.
\item
  \textbf{Niels W. Hanson}, W. Evan Durno, Kishori M. Konwar, Steven J.
  Hallam. \emph{IMPROV: An integrated MetaPRO-teomics viewer.} Eleventh Asia Pacific Bioinformatics Conference (APBC 2013). Poster. Vancouver, BC, Canada. 21-24 January 2013
\item \textbf{Niels W. Hanson}, Jody J. Wright, Kishori M. Konwar, Steven J. Hallam. International Symposium on Microbial Ecology (ISME14). Poster. Copenhagen, Denmark. August 19--24 2012.
\item
  \textbf{Niels W. Hanson}, Antoine P. Pag\'{e}, Kishori M. Konwar, Charles
  G. Howes, Steven J. Hallam. \emph{Metabolic Interaction Networks for
  the Whole Community}. Tenth Asia Pacific Bioinformatics Conference (APBC 2012). Poster. Melbourne, Australia. January 17-19 2012.
\end{itemize}

\section{Journal \& Conference Reviewing}
\begin{itemize}
\item 2015 IEEE Conference on Computational Intelligence in Bioinformatics and Computational Biology (CIBCB 2015)
\item 11th International Symposium on Bioinformatics Research and Applications (ISBRA 2015)
\item 2014 IEEE Conference on Computational Intelligence in Bioinformatics and Computational Biology (CIBCB 2014)
\end{itemize}

\section{Teaching}

\begin{entrylist}
\entry
{Aug. 2014}
{Marine Biological Laboratory}
{STAMPS Course}
{\emph{Teaching Assistant} \\
Teaching Assistant for the prestigious Strategies and Techniques for Analyzing Microbial Population Structure (STAMPS) course offered at the Marine Biological Laboratory (MBL) in Woods Hole, Massachusetts. Lectured and conducted tutorials to more than 60 post-docs, graduate students, and professors on the use of MetaPathways to process and annotate environmental sequence information, and downstream statistical analysis and visualization using the R, ggplot2, dplyr, and Pathway Tools software.}
\end{entrylist}
\begin{entrylist}
\entry
{2014}
{University of British Columbia}
{Problem Based Learning in Bioinformatics (BIOF 520)}
{\emph{Teaching Assistant} \\
Graduate teaching assistant of the Problem Based Learning in Bioinformatics (BIOF 520) graduate class. Responsibilities included assisting with lectures, designing and facilitating the Human Microbiome PBL learning module, grading assignments, and general class administration.}
\end{entrylist}
\begin{entrylist}
\entry
{2013--2014}
{University of British Columbia}
{Ecological Genomics (MICB 425)}
{\emph{Teaching Assistant} \\
Conducted 1--2 weeks of lectures annually in the Microbial
  Ecological Genomics class (MICB 425). Topics included the
  functional and taxonomic analysis of environmental sequence
  information, with a focus on using the MetaPathways pipeline. Analytical topics
  included down-stream statistical analysis including Hierarchical
  Clustering, Principal Component Analysis (PCA), Non-metric
  Multidimensional Scaling (NMDS), and general visualization in the R
  statistical environment using ggplot2 and reproducible analysis using RMarkdown.}
\end{entrylist}

\section{Projects}

\begin{itemize}
\item De-Confusion Tables. An interactive R shiny app for exploring performance statistics of a binary classifier. Final project. Developing Data Products. John Hopkins University Coursera Class. July 2014. \href{http://goo.gl/Fl78Qh}{http://goo.gl/Fl78Qh}
\item D3.js visualizations:
\begin{itemize}
	\item World Map: World map displaying metagenomic sequencing data. \href{http://goo.gl/WVjdAv}{http://goo.gl/WVjdAv}
	\item Sunburst Plot: Radial tree-map showing hierarchal classification of global metagenomes. \href{http://goo.gl/3LlXIq}{http://goo.gl/3LlXIq}
	\item Bubble Tree: Modified dendrogram to show taxonomy across multiple samples. \href{http://goo.gl/2SBVsp}{http://goo.gl/2SBVsp}
	\item Bubble Plot: A two-variable sortable bubble plot. \href{http://goo.gl/Jfslu1}{http://goo.gl/Jfslu1}
	\item Heatmap: Two-variable heatmap with calculated marginal distributions. \href{http://goo.gl/3dIurm}{http://goo.gl/3dIurm}
\end{itemize}
\item FastLCA R-Package. An implementation the FastLCA method for local correlations in R. Utilizes the optimized C code and accepts Data Frames from R. Parallelized using Open-MP. \href{http://goo.gl/cPb1uP}{http://goo.gl/cPb1uP}
\end{itemize}

%----------------------------------------------------------------------------------------
%	INTERESTS SECTION
%----------------------------------------------------------------------------------------

\section{Interests}

\textbf{Professional:} Data Analysis, Information Visualization, Machine Learning, High-performance Computing, Web-app Creation, Software Design  \textbf{Personal:} Running, Tennis, Badminton, Piano, Photography, Cooking 

\section{References}

Provided upon request.

\end{document}